\documentclass{zirkelblatt}
\usepackage{tikz}
\begin{document}

\maketitle{Klasse 11./12., Gruppe 2}{30. November 2013}

\begin{aufgabe}{Goldener Schnitt}
Der \emph{goldene Schnitt} ist die positive Lösung der quadratischen Gleichung
\[ \Phi^2 - \Phi - 1 = 0. \]
\begin{enumerate}
\item Stelle~$\Phi$ als Wurzelausdruck dar.
\item Weise folgende kuriose Beziehungen nach:
$\Phi = 1 + \frac{1}{\Phi}$,$\ $
$(1/\Phi)^2 = 1 - (1/\Phi)$.
\item Informiere dich über geometrische Interpretationen von~$\Phi$.
\end{enumerate}
\end{aufgabe}

\begin{aufgabe}{Geometrische Reihe}
\begin{enumerate}
\item Beweise folgende Summenformel für Potenzen einer beliebigen Zahl~$q$:
\[ q^0 + q^1 + q^2 + \cdots + q^{n-2} + q^{n-1} + q^n = \frac{1 - q^{n+1}}{1 -
q}. \]
\emph{Tipp:} Multipliziere beide Seiten mit~$(1-q)$.

\item Sei~$0 \leq q < 1$. Mache folgende Formel über eine \emph{unendliche
Summe} plausibel:
\[ q^0 + q^1 + q^2 + \cdots = \frac{1}{1 - q}. \]

\item Gilt diese Formel auch, wenn~$-1 < q \leq 0$?

\item Was passiert bei~$q \geq 1$ oder $q \leq -1$?

\item Berechne~$q^2 + q^3 + \cdots$ im Spezialfall~$q := 1/\Phi$.
{(Konventionsgemäß gilt~$q^0 := 1$.)}
\end{enumerate}
\end{aufgabe}

%\begin{aufgabe}{Fibonaccizahlen}
%Die Folge der \emph{Fibonaccizahlen} wird für~$n \geq 2$ rekursiv durch
%\[ f_{n+1} := f_n + f_{n-1} \]
%definiert, mit den Anfangswerten~$f_1 := 1$ und~$f_2 := 1$.
%\begin{enumerate}
%\item Bestimme so viele Fibonaccizahlen, wie du möchtest.
%\item Zeige, dass der Quotient~$f_{n+1}/f_n$ zweier aufeinander folgender Fibonaccizahlen
%folgende Beziehung erfüllt:
%\[ \frac{f_{n+1}}{f_n} = 1 + \frac{f_{n-1}}{f_n}. \]
%\item Man kann beweisen, dass die Folge der Quotienten einen Grenzwert besitzt:
%\[ \lim_{n \to \infty} \frac{f_{n+1}}{f_n} =: q. \]
%Beweise, dass dieser Grenzwert gleich~$\Phi$ ist.
%\end{enumerate}
%\end{aufgabe}

\begin{aufgabe}{Conways Armee}
Ein unendlich ausgedehntes Damebrett sei in zwei Hälften zerteilt (siehe
Rückseite). Im unteren Teil darf man beliebig viele Damesteine platzieren. Ziel
des Spiels ist es, einen Damestein möglichst hoch in das obere Spielfeld zu
bringen. Dabei darf nur folgender Spielzug angewendet werden: Ein Stein
darf einen (horizontal oder vertikal) benachbarten Stein überspringen, wenn das
Zielfeld unbesetzt ist. Der übersprungene Stein wird dann aus dem Spiel
entfernt.
\begin{enumerate}
\item Mit wie vielen Steinen in der unteren Bretthälfte muss man beginnen, um
Höhe~1, Höhe~2, Höhe~3 oder Höhe~4 über der Trennlinie zu erreichen?
\end{enumerate}
Wenn ein Stein die
"`Taxi"'-Entfernung~$n$ zum angepeilten Zielstein hat, weisen wir ihm den \emph{Wert}~$x^n$ zu.
Dabei ist~$x$ eine Konstante, die wir später festlegen werden.
Der angepeilte Zielstein selbst hat also den Wert~$x^0 = 1$, seine vier
unmittelbaren Nachbarn haben den Wert~$x^1 = x$.
\begin{enumerate}
\addtocounter{enumi}{1}
\item Wie ändert sich die Summe der Steinwerte bei einem Zug? (Unterscheide drei Arten von Sprüngen.)
\end{enumerate}
Wir setzen nun~$x := 1/\Phi$, wobei~$\Phi$ der goldene Schnitt aus Aufgabe~1
ist.
\begin{enumerate}
\item[$\star$ c)] Berechne den Gesamtwert einer \emph{vollständig mit
Steinen besetzten Zeile}, die sich eine Ebene unterhalb des Zielsteins befindet.
\item[$\star$ d)] Was ist der Gesamtwert einer \emph{vollständig besetzten unteren Bretthälfte}?
\item[$\star$ e)] Folgere: Höhe~$5$ ist nur mit unendlich vielen Steinen
erreichbar.
\item[$\star$ f)] Vielleicht möchtest du Conways Armee im Computer implementieren.
\end{enumerate}
\end{aufgabe}

\newpage

\newcommand*{\xMin}{0}%
\newcommand*{\xMax}{6}%
\newcommand*{\yMin}{0}%
\newcommand*{\yMax}{6}%
\hspace*{-1.8cm}%
\begin{tikzpicture}
    \draw[step=2cm,gray,very thin] (-0.3,-0.3) grid (18.3,26.3);
    \draw[thick] (8,-0.3) -- (8,26.3);
\end{tikzpicture}

\end{document}
