\documentclass{zirkelblatt}
\geometry{tmargin=2cm,bmargin=2cm,lmargin=3cm,rmargin=3cm}
\begin{document}

\maketitle{Klasse 11./12., Gruppe 2}{21. Dezember 2013}

\begin{block}{Konventionen}
\begin{itemize}
\item[]
Die \emph{Menge der natürlichen Zahlen} ist~$\NN := \{ 0, 1, 2, 3, \ldots \}$.
\item[]
Die \emph{leere Menge} $\{ \} = \emptyset$ enthält kein einziges Element.
\item[]
Alle Elemente der Menge der Elefanten in diesem Raum können~$\pi$ auswendig.
\end{itemize}
\end{block}

\begin{block}{Regeln für surreale Zahlen}
\renewcommand{\labelenumi}{\arabic{enumi}.}
\begin{enumerate}
\item \emph{Konstruktionsprinzip.}
Sind~$L$ und~$R$ Mengen surrealer Zahlen und \hil{ist kein Element von~$L$
$\geq$ irgendeinem Element von~$R$}, so ist~$\sur{L}{R}$ ebenfalls eine surreale
Zahl. Alle surrealen Zahlen entstehen auf diese Art.

\item \emph{Notation.}
Für~$x = \sur{L}{R}$ bezeichnen wir ein typisches Element von~$L$
mit~"`$x^L$"', ein typisches Element von~$R$ mit~"`$x^R$"'. Wenn
wir~"`$\sur{a,b,c,\ldots}{d,e,f,\ldots}$"' schreiben, meinen wir die
Zahl~$\sur{L}{R}$, sodass~$a,b,c,\ldots$ die typischen Elemente von~$L$
und~$d,e,f,\ldots$ die typischen Elemente von~$R$ sind.

\item \emph{Anordnung.}

Wir sagen genau dann~$x \geq y$, falls kein $x^R \leq y$ und~$x \leq$
keinem $y^L$.

Wir sagen genau dann~$x \not\leq y$, wenn~$x \leq y$ nicht gilt.

Wir sagen genau dann~$x < y$, wenn $x \leq y$ und~$y \not\leq x$.

Wir sagen genau dann~$x \leq y$, wenn~$y \geq x$.

Wir sagen genau dann~$x > y$, wenn~$y < x$.

\item \emph{Gleichheit.}
Wir sagen genau dann~$x = y$, wenn~$x \leq y$ und~$y \leq x$.

\item \emph{Addition.} $x + y := \sur{x^L + y, x + y^L}{x^R + y, x + y^R}$.

\item \emph{Negation.} $-x := \sur{-x^R}{-x^L}$.

\item \emph{Multiplikation.}
$xy = \sur{x^Ly + xy^L - x^Ly^L, x^Ry + xy^R - x^Ry^R}{x^Ly + xy^R -
x^Ly^R, x^Ry + xy^L - x^Ry^L}$.
\end{enumerate}
\end{block}

\newpage

\begin{aufgabe}{Erste Beispiele für surreale Zahlen}
Zu Beginn ist uns keine einzige surreale Zahl bekannt. Trotzdem kennen wir
eine \emph{Menge} surrealer Zahlen: nämlich die leere Menge. So können wir nach
dem Konstruktionsprinzip eine erste surreale Zahl bauen:
\begin{align*}
  0 &:= \sur{}{} \quad\text{(also $L = R = \emptyset$)} \\
\intertext{Wir haben diese Zahl~"`$0$"' genannt, weil sie die Rolle der Null einnehmen
wird. Mit dieser Zahl an der Hand können wir eine weitere surreale Zahl bauen:}
  1 &:= \sur{0}{} \quad\text{(also $L = \{ 0 \}, R = \emptyset$)}
\end{align*}

\begin{enumerate}
\item Überzeuge dich davon, dass die so definierten Zahlen~$0$ und~$1$ wirklich
surreale Zahlen sind, dass also die \hil{Voraussetzung} in der Konstruktionsvorschrift
jeweils erfüllt war.

\item Überprüfe, dass gemäß der Definitionen tatsächlich~$0 \leq 1$ gilt.

\item Mit der bereits konstruierten Zahl~$0$ kann man insgesamt drei Ausdrücke
angeben:
\[ \sur{0}{}, \quad \sur{}{0}, \quad \sur{0}{0}. \]
Welche der beiden hinteren Ausdrücke sind Zahlen?

\item Sortiere alle bis jetzt gefundenen Zahlen und überlege dir so geeignete
Bezeichnungen für die neuen Zahlen aus~c).

\item Konstruiere ein paar weitere Zahlen, sortiere sie in die bereits
gefundenen Zahlen ein und überlege dir geeignete Namen für sie.
\end{enumerate}
\end{aufgabe}

\begin{aufgabe}{Erste Rechnungen mit surrealen Zahlen (benötigt Aufgabe 1)}
\begin{enumerate}
\item Überprüfe, dass gemäß der Definitionen gilt: $0 + 1 = 1$.
\item Berechne~$(-1) + 1$ und vergleiche das Ergebnis mit~$0$.
\end{enumerate}
\end{aufgabe}

\begin{aufgabe}{Mex-Operation}
Ist~$S$ eine endliche Menge natürlicher Zahlen, so ist~$\mex S$ die \emph{kleinste}
natürliche Zahl, die \emph{nicht} in~$S$ liegt (minimum excludant).
\begin{enumerate}
\item Überzeuge dich von der Richtigkeit folgender Beispiele:
\[
  \mex \{ 0,1,4,7 \} = 2, \quad
  \mex \{ 1,4,7 \} = 0, \quad
  \mex \emptyset = 0. \]
\item Berechne das Mex von deiner Lieblingsteilmenge natürlicher Zahlen.
\end{enumerate}
\end{aufgabe}

\begin{aufgabe}{Nimber-Addition (benötigt Aufgabe 3)}
Die \emph{Nimber-Addition} ist in mengentheoretischer Notation wie folgt rekursiv definiert:
\[ n \oplus m := \mex\Bigl(\left\{n' \oplus m \,|\, n' < n\right\} \cup
\left\{n \oplus m' \,|\, m' < m\right\}\Bigr). \]
Wenn man also den Wert von~$n \oplus m$ herausfinden möchte, muss man zunächst
die Werte von~$n' \oplus m$ für alle kleineren Zahlen~$n' < n$ und die Werte
von~$n \oplus m'$ für alle kleineren Zahlen~$m' < m$ bestimmen. Der Wert
von~$n \oplus m$ ergibt sich dann als Mex dieser Zahlen.
\begin{enumerate}
\item Ergänze unten stehende Tabelle für die Nim-Addition.
\item[$\star$ b)] Wenn du schon die Beweistechnik der Induktion kennst, kannst
du dich an folgenden Behauptungen für alle~$n \in \NN$ versuchen:
\begin{align*}
  0 \oplus n &= n \\
  n \oplus n &= 0
\end{align*}
\end{enumerate}
\begin{center}
  \begin{tabular}{r|ccccccccl}
    $n \setminus m$ & 0 & 1 & 2 & 3 & 4 & 5 & 6 & 7 & $\cdots$ \\\hline
    0 & 0 & 1 & 2 \\
    1 &   & 0 & 3 \\
    2 & \\
    3 & & & & & & & 4 \\
    4 & & & 6 \\
    5 & \\
    6 & \\
    7 & & & & & & 2 \\
    \vdots
  \end{tabular}
\end{center}
\end{aufgabe}

\begin{aufgabe}{Falsche binomische Formel}
\ldots wäre schön, benötigt aber Multiplikation; hat daher hohen technischen
Aufwand.
\end{aufgabe}

\end{document}
