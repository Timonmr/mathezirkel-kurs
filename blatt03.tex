\documentclass{zirkelblatt}
\usepackage{young}
\geometry{tmargin=2cm,bmargin=3cm,lmargin=3cm,rmargin=3cm}
\usepackage{lscape}
\newcommand{\head}[1]{\section*{\rmfamily #1}}%begin{center}\large \textbf{#1}\end{center}}
\let\raggedsection\centering
\newcommand{\fuzzy}{\mathrel{||}}
\newcommand{\ol}[1]{\ensuremath{\overline{#1}}}
\newcommand{\ZZ}{\mathbb{Z}}

%\setkeys{Gin}{draft}
\begin{document}

\maketitle{Klasse 11./12., Gruppe 2}{15. März 2014}

% Konventionen: ganze Zahlen
% teilerfremd <==> ggT=1.
% 0 teilerfremd

\head{Zahlentheoretische Grundlagen}

Ein \emph{Ring} ist eine algebraische Struktur mit einer Addition und
Multiplikation, die mit der gewöhnlichen Addition und Multiplikation von ganzen
Zahlen zu tun haben können, aber nicht unbedingt müssen. Von einem Ring fordert
man die folgenden Axiome:
\begin{align*}
    0 + x &= x = x + 0 \\
    x + y &= y + x \\
    x + (y + z) &= (x + y) + z \\\\
    1 \cdot x &= x = x \cdot 1 \\
    x \cdot y &= y \cdot x \\
    x \cdot (y \cdot z) &= (x \cdot y) \cdot z \\\\
    x \cdot (y + z) &= x \cdot y + x \cdot z \\
    (x + y) \cdot z &= x \cdot z + y \cdot z
\end{align*}
Dabei bezeichnen die Symbole~"`$0$"' und~"`$1$"' zwei besondere Elemente des
Rings, nicht unbedingt die bekannten Zahlen Null und Eins. Man nennt sie wegen
ihrer besonderen Bedeutung \emph{Null-} und \emph{Einselement}. Ein weiteres
Axiom besagt, dass es zu jedem Element~$x$ ein weiteres Element geben soll,
bezeichnet~$-x$, das bezüglich der Addition zu $x$ ist:
\[ x + (-x) = 0 = (-x) + x. \]
Weitere Axiome fordert man von Ringen nicht.

\begin{aufgabe}{Beispiele und Nichtbeispiele für Ringe}
Mache dir klar:
\begin{enumerate}
\item Die rationalen Zahlen bilden bezüglich der gewöhnlichen
Addition und Multiplikation einen Ring.
\item Die ganzen Zahlen bilden bezüglich der gewöhnlichen
Addition und Multiplikation einen Ring.
\item Die natürlichen Zahlen bilden bezüglich der gewöhnlichen Addition und
Multiplikation \emph{keinen} Ring.
\end{enumerate}
Für die Kryptographie gehören die \emph{Restklassenringe} zu den wichtigsten
Ringen.
\end{aufgabe}

\newpage
\begin{aufgabe}{Restklassenarithmetik}
Sei~$m$ eine feste positive Zahl. Dann besteht der
\emph{Restklassenring}~$\ZZ/(m)$ (oft auch~"`$\ZZ_m$"' geschrieben) aus den
verschiedenen Resten, die bei Division durch~$m$ auftreten können:
\[ \ZZ/(m) = \{ \ol{0}, \ol{1}, \ol{2}, \ldots, \ol{m-1} \}. \]
Wenn man faul ist, lässt man die Oberstriche auch weg. In~$\ZZ/(m)$ addiert und
multipliziert man fast wie gewohnt -- nur dass man nach jedem Rechenschritt die
Ergebnisse \emph{modulo~$m$} vereinfachen kann.
\begin{enumerate}
\item Überzeuge dich, dass in~$\ZZ/(6)$ folgende Rechnung stimmt:
\[ \ol{4} + \ol{7} = \ol{11} = \ol{5}. \]
\item Ergänze unten stehende Tabellen für die Addition und Multiplikation
in~$\ZZ/(4)$.
\item Ein Element~$x$ eines Rings heißt genau dann \emph{invertierbar}, wenn es
ein weiteres Element~$y$ mit der Eigenschaft~$xy = 1$ gibt. Das Element~$y$
heißt dann auch \emph{Inverses} von~$x$. Welche Elemente von~$\ZZ/(4)$ sind
invertierbar?
\end{enumerate}
\begin{center}
  \begin{tabular}{r|cccc}
    $+$    & \ol{0} & \ol{1} & \ol{2} & \ol{3} \\\hline
    \ol{0} & \ol{0} & \ol{1} & \ol{2} \\
    \ol{1} &   &        &        \\
    \ol{2} & \\
    \ol{3} & & & \ol{1}
  \end{tabular}
  \qquad
  \begin{tabular}{r|cccc}
    $\cdot$ & \ol{0} & \ol{1} & \ol{2} & \ol{3} \\\hline
    \ol{0} & \ol{0} & \ol{0} & \ol{0} \\
    \ol{1} &   &        &        \\
    \ol{2} & \\
    \ol{3} & & & \ol{2}
  \end{tabular}
\end{center}
\end{aufgabe}

\begin{aufgabe}{Falsche binomische Formel}
Für ganze Zahlen~$x,y$ gilt bekanntermaßen die \emph{binomische Formel}:
\[ (x+y)^2 = x^2 + 2xy + y^2. \]
\begin{enumerate}
\item Beweise diese Formel rechnerisch.
\item Gib einen geometrischen Beweis, der etwas mit Quadraten der
Seitenlängen~$x$ und~$y$ zu tun hat.
\item Beweise, dass die binomische Formel sogar in jedem Ring gilt. Dabei ist allgemein~"`$a^2$"'
eine Abkürzung für~$a \cdot a$ und~"`$2$"' eine Abkürzung für~$1 + 1$ (was auch
immer das in dem untersuchten Ring ergeben mag).
\item Beweise, dass im Ring~$\ZZ/(2)$ außerdem die sogenannte
\emph{falsche binomische Formel} gilt:
\[ (x+y)^2 = x^2 + y^2. \]
\item Wenn du aus der Informatik die logischen Gatter (wie UND, ODER, \ldots)
kennst, interessiert dich vielleicht folgende Frage: Was haben die Addition und
die Multiplikation von~$\ZZ/(2)$ mit den logischen Gattern zu tun?
\end{enumerate}
\end{aufgabe}

\newpage
\begin{aufgabe}{Euklidischer Algorithmus}
Eines der ältesten überlieferten numerischen Verfahren ist der
\emph{euklidische Algorithmus}. Mit seiner Hilfe kann man auf effiziente Art und Weise
den größten gemeinsamen Teiler zweier ganzer Zahlen bestimmen -- viel
schneller, als wenn man erst die Zahlen in Primfaktoren zerlegen würde.
\begin{align*}
    42 &= 1 \cdot 26 + 16 \\
    26 &= 1 \cdot 16 + 10 \\
    16 &= 1 \cdot 10 + 6 \\
    10 &= 1 \cdot 6 + 4 \\
    6 &= 1 \cdot 4 + 2 \\
    4 &= 2 \cdot 2 + 0
\end{align*}
\begin{enumerate}
\item In diesem Beispiel wurde der euklidische Algorithmus verwendet, um den
größten gemeinsamen Teiler von~$42$ und~$26$ zu ermitteln (dieser ist~$2$).
Erschließe, wie das Verfahren funktioniert.
\item Bestimme mit dem euklidischen Algorithmus den größten gemeinsamen Teiler
zweier Zahlen deiner Wahl.
\item Das Verfahren hört dann auf, wenn als Rest~$0$ auftritt. Erkläre, wieso
das unabhängig von den Anfangszahlen stets nach endlich vielen Schritten der
Fall ist! (Man sagt, dass der euklidische Algorithmus \emph{terminiert}.)
\item Mit dem Algorithmus kann man durch \emph{Rückwärtsauflösen} den größten
gemeinsamen Teiler~$d$ zweier ganzer Zahlen~$x$ und~$y$ in der Form~$d = ax +
by$ für gewisse Hilfszahlen~$a$ und~$b$ schreiben. Versuche das in obigem
Beispiel!
\end{enumerate}
Eine Darstellung der Form~$d = ax + by$ des größten gemeinsamen Teilers heißt
auch \emph{Bézoutdarstellung}. Es ist etwas sehr besonderes, dass es im Ring
der ganzen Zahlen eine solche immer gibt.
\end{aufgabe}

\begin{aufgabe}{Invertierbarkeit in Restklassenringen}
In den Restklassenringen ist nicht jedes Element invertierbar, das haben wir
schon beim Beispiel mit~$\ZZ/(4)$ gesehen. Es gilt folgende Regel: Ein
Element~$\ol{a}$ von~$\ZZ/(m)$ ist genau dann invertierbar, wenn die Zahlen~$a$
und~$m$ zueinander teilerfremd sind.
\begin{enumerate}
\item Bestätige diese Regel im Beispiel~$\ZZ/(4)$.
\item Erkläre, wie man den euklidischen Algorithmus verwenden kann, um
in~$\ZZ/(m)$ Inverse zu berechnen!
\item Welche Elemente sind in~$\ZZ/(p)$ invertierbar, wenn~$p$ eine Primzahl
ist?
\end{enumerate}
\end{aufgabe}

\newpage
\begin{aufgabe}{Eulersche Phi-Funktion}
Die Anzahl der in~$\ZZ/(m)$ invertierbaren Elemente schreibt man
auch~"`$\Phi(m)$"'. Sei im Folgenden~$p$ eine beliebige Primzahl.
\begin{enumerate}
\item Zeige: $\Phi(p) = p - 1$.
\item Zeige: $\Phi(p^2) = p^2 - p$.
\item Was ist~$\Phi(p^3)$?
\item Was ist~$\Phi(p^n)$?
\end{enumerate}
Später werden wir verstehen, dass für teilerfremde Zahlen~$a,b$ die Rechenregel
\[ \Phi(ab) = \Phi(a) \cdot \Phi(b) \]
gilt. Damit kann man die Werte der eulerschen Phi-Funktion recht effizient
berechnen.
\end{aufgabe}

\end{document}

Chinesischer Restsatz?
