\documentclass[a4paper,ngerman]{scrartcl}

%\usepackage{ucs}
\usepackage[utf8]{inputenc}

\usepackage[ngerman]{babel}

\usepackage{amsmath,amsthm,amssymb,amscd,color,graphicx}

%\usepackage[small,nohug]{diagrams}
%\diagramstyle[labelstyle=\scriptstyle]

\usepackage[protrusion=true,expansion=true]{microtype}

\usepackage{lmodern}
\usepackage{tabto}

\usepackage[natbib=true,style=numeric]{biblatex}
\usepackage[babel]{csquotes}
\bibliography{lit}

\usepackage[all]{xy}

\usepackage{hyperref}

\setlength\parskip{\medskipamount}
\setlength\parindent{0pt}

\clubpenalty=10000
\widowpenalty=10000
\displaywidowpenalty=10000

%\newarrow{Equals}=====

%\usepackage{geometry}
%\geometry{tmargin=2cm,bmargin=4cm,lmargin=3cm,rmargin=3cm}

\renewcommand{\labelitemi}{--}

\begin{document}

\section*{Fraktale}

\begin{itemize}
\item
Interaktive Fraktalexploration mit Computerunterstützung:
Schüler können interessante und ästhetische Bereiche etwa des
Mandelbrotfraktals suchen~\cite{fraktale}.
\item
Kochsche Schneeflocke~\cite{koch}: Schüler können die Kochsche Schneeflocke
Schritt für Schritt konstruieren und einsehen, dass ihr Umfang paradoxerweise
unendlich ist.
\item
Länge von Küstenlinien: Man könnte Landkarten derselben Küste, aber
verschiedenen Maßstabs austeilen und die Schüler die scheinbaren Längen
bestimmen lassen. Verblüffenderweise werden die Antworten dann jeweils
unterschiedlich ausfallen.
\item
Chaosspiel~\cite{chaosspiel}, um das Sierpinskidreieck zu konstruieren, mit
Com\-pu\-ter\-un\-ter\-stüt\-zung: Zunächst lässt man den Computer die Ecken zufällig
bestimmen. Das Ergebnis wird das Sierpinskidreieck sein, allerdings werden
uns die Schüler vielleicht Manipulation unterstellen. Dann sollen sie selbst
die Zufallsecken vorgeben, um jede Einmischung unsererseits auszuschließen.
Verblüffenderweise wird sich wieder Ordnung aus dem Chaos ergeben.
\item Labyrinthe: Schüler können auf systematische Art und Weise komplizierte
Labyrinthe zeichnen, die dann fraktaler Natur sind.
\end{itemize}
Botschaften: Aus Zufall kann Ordnung entstehen; Dinge sind manchmal nicht so,
wie sie scheinen; auch einfache Objekte können faszinierende mathematische
Eigenschaften haben; Mathematik kann ästhetisch sein.


\section*{Kryptographie und Steganographie}

\begin{itemize}
\item Cäsar-Chiffre und allgemeine Substitutionschiffren~\cite{subst}: Schüler können Texte
gegenseitig verschlüsseln und sich an ihrer Entschlüsselung (über bekannte
Buchstabenhäufigkeitsverteilung) versuchen.
\item Steganographie: Schüler können gegenseitig kurze Botschaften in größeren Texten
verstecken (etwa über die Anfangsbuchstaben jeder Zeile) und sich an ihrer Enttarnung
versuchen. Kann mit Musik kombiniert werden, indem man Botschaften über
Notennamen kodiert.
\item Idee hinter Public-Key-Kryptographie: Wie kann Alice an Bob ein
wertvolles Paket per Post schicken, ohne dass Briefträger Zugriff auf den
Inhalt erhalten können? Schüler können verschiedene Strategien dazu entwickeln
und auf ihre Sicherheit untersuchen.
\end{itemize}
Botschaften: Datensicherheit ist wichtig; Mathematik ist in der Praxis (und
nicht nur im Supermarkt) nützlich; Mathematik kann sehr kreativ sein.


\section*{Dynamische Labyrinthe}

Wir könnten den erprobten Baukasten \emph{Dynamische Labyrinthe} nach Inge
Schwank~\cite{laby} einsetzen:
Schüler könnten verschiedene Schaltpläne testen und selbst erfinden. Damit
würde man Sensibilität dafür, wie Computer arbeiten, wecken; spielerischen
Umgang mit Programmierung zeigen; und kreatives Potenzial nutzen.


\section*{Rechenverfahren der Antike}

Hier könnte man das Rechnen mit
\begin{itemize}
\item Abakus,
\item Rechenschieber,
\item römischen Zahlen,
\item usw.
\end{itemize}
üben.

Botschaften: Die geschickte Darstellung von Zahlen kann das Rechnen erheblich
vereinfachen; schon mit einfachen Hilfsmitteln kann man effizient mit großen
Zahlen arbeiten; Mathematik ist ein Kulturgut.


\section*{Magische Quadrate}

\begin{itemize}
\item Man könnte Schüler zunächst an kleinen Quadraten knobeln lassen.
\item Dann können sie wertschätzen, dass es Algorithmen gibt, um systematisch
magische Quadrate aufzustellen.
\item Die Unterraumeigenschaften magischer Quadrate könnte man (entsprechend
aufbereitet) diskutieren.
\end{itemize}

Botschaften: Mathematik hilft, um Rätsel zu lösen und allgemeiner Knobeln durch
systematische Lösungsversuche zu ersetzen bzw. zu ergänzen.

\renewcommand\refname{Internetverweise}
\nocite{*}
\printbibliography

\end{document}
