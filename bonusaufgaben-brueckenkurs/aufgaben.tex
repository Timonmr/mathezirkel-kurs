\documentclass{../zirkelblatt1415}

\let\raggedsection\centering

\definecolor{shadecolor}{rgb}{1,1,1}

\begin{document}

\pagestyle{empty}

\vspace*{-2cm}
\enlargethispage{2cm}

\section*{Gelangweilt?}

\begin{aufgabe}{Der binomische Lehrsatz}
Zeige mit Induktion über~$n$, dass für alle Zahlen~$a$ und~$b$ gilt:
\[ (a + b)^n =
  \binom{n}{0} a^n +
  \binom{n}{1} a^{n-1} b +
  \binom{n}{2} a^{n-2} b^2 + \cdots +
  \binom{n}{n-1} a b^{n-1} +
  \binom{n}{n} b^n. \]
\emph{Tipp:} $\binom{n+1}{k} = \binom{n}{k-1} + \binom{n}{k}$.
\end{aufgabe}

\begin{aufgabe}{Kettenbruchentwicklungen}
Der \emph{goldene Schnitt} ist die Zahl~$\Phi = \frac{1 + \sqrt{5}}{2}$. Als
Teilungsverhältnis kommt~$\Phi$ an erstaunlich vielen Orten in der Natur und in
der Kunst vor. Zeige:
\[ \Phi = 1 + \cfrac{1}{1 + \cfrac{1}{1 + \cfrac{1}{1 + \ddots}}},
  \qquad\qquad\qquad
  \sqrt{2} = 1 + \cfrac{1}{2 + \cfrac{1}{2 + \cfrac{1}{2 + \ddots}}}. \]
\emph{Hinweis:} Du darfst voraussetzen, dass die unendlichen Kettenbrüche
überhaupt Sinn ergeben, d.\,h. konvergieren. Wie man das beweist, lernt man in
Analysis~I.
\end{aufgabe}

\begin{aufgabe}{Die 10-adischen Zahlen}
Bei den gewöhnlichen reellen Zahlen stehen in ihrer Dezimalschreibweise vor dem
Komma nur endlich viele Ziffern, hinter dem Komma aber gelegentlich unendlich
viele Ziffern. Bei den~$10$-adischen Zahlen ist es genau umgekehrt: Vor dem
Komma dürfen unendlich viele Ziffern stehen, hinter dem Komma dagegen nur
endlich viele. Die Rechenverfahren zur Addition, Subtraktion und
Multiplikation, wie man sie aus der Schule kennt, funktionieren weitestgehend
unverändert. Die Division wird etwas komplizierter.
\begin{enumerate}
\item Vollziehe folgende Rechnung nach:
$\ldots 13562 + \ldots 90081 = \ldots 03643$.
\item Was ist $\ldots 99999 + 1$? Dabei ist~$1 = \ldots 00001$.
\item Was ist~$-123$?
\item Finde eine~$10$-adische Zahl~$x$ -- weder Null noch Eins -- mit~$x^2 = x$.
\end{enumerate}
{\scriptsize
\emph{Bemerkung:} Die Gleichung in Teilaufgabe~d) kann man zu~$x \cdot (x-1) =
0$ umstellen. In den~$10$-adischen Zahlen kann also ein Produkt Null sein, ohne
dass einer der Faktoren Null ist. Wegen dieser schlechten Eigenschaft werden
die~$10$-adischen Zahlen kaum verwendet. \emph{Allerdings:} Verwendet man als
Basis nicht~$10$, sondern eine Primzahl, so gibt es dieses Problem nicht.
Die~$2$-adischen Zahlen werden gelegentlich in der Informatik und
die~$p$-adischen Zahlen, wobei~$p$ irgendeine Primzahl ist, überall in der
Zahlentheorie verwendet. Dort gibt es beispielsweise folgendes mächtiges
"`lokal-zu-global"' Prinzip: Eine Gleichung einer bestimmten Art hat genau dann
eine Lösung in~$\mathbb{Z}$, wenn sie eine Lösung in~$\mathbb{R}$ und jeweils
eine Lösung in allen~$p$-adischen Zahlen hat.\par}
\end{aufgabe}

\begin{center}
  \emph{Lösungsbesprechung individuell per Mail (iblech@web.de) oder \\
  jederzeit persönlich in Büro 2031/L1}
\end{center}

\end{document}
