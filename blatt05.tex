\documentclass{zirkelblatt}
\geometry{tmargin=2cm,bmargin=3cm,lmargin=3cm,rmargin=3cm}
\usepackage{empheq}
\usepackage{float}
\floatstyle{ruled}
\restylefloat{figure}
\newcommand{\head}[1]{\section*{\rmfamily #1}}%begin{center}\large \textbf{#1}\end{center}}
\let\raggedsection\centering
\newcommand{\ZZ}{\mathbb{Z}}
\newcommand{\RR}{\mathbb{R}}

\begin{document}

\maketitleSF{Klasse 11./12., Gruppe 2}{31. Mai 2014: \\ Synthetische
Differentialgeometrie}

\section{Einleitung}

Eine \emph{infinitesimale Zahl}~$\varepsilon$ ist eine Zahl, deren Quadrat Null
ist:~$\varepsilon^2 = 0$. In der gewöhnlichen mathematischen Welt gibt es nur
eine einzige infinitesimale Zahl, nämlich die Zahl~$0$. Für gewisse Anwendungen wäre es
aber schön, wenn es auch interessantere infinitesimale Zahlen gäbe. Ähnlich wie
bei den komplexen Zahlen kann man solche Zahlen \emph{künstlich} konstruieren;
anders als bei den komplexen Zahlen muss man dafür aber gleich ein ganzes neues
\emph{mathematisches Universum} errichten.

Das Teilgebiet der Mathematik, in dem man solche Universen studiert, heißt
\emph{synthetische Differentialgeometrie} und ist ein relativ junges
Forschungsgebiet. Erste Ideen gehen auf die alten Griechen und Versuche des
dänischen Mathematikers Johannes Hjelmslev zurück (*~1873, †~1950), richtig
initiiert wurde das Gebiet aber erst in den 1970er Jahren durch Anders Kock,
ebenfalls Däne.

Im Folgenden werden wir lernen, wozu infinitesimale Zahlen nützlich sind; wie
man in dem alternativen Universum arbeiten kann; und schließlich inwieweit
Resultate, die man in der neuen mathematischen Welt erzielt, auch in der
gewöhnlichen mathematischen Welt Gültigkeit haben. Ohne diesen letzten Punkt
wäre unser Unterfangen ein reines Gedankenexperiment ohne Nutzen.

Dreh- und Angelpunkt für synthetische Differentialgeometrie ist ein Axiom,
das \emph{klassisch} -- das heißt im gewöhnlichen mathematischen Universum --
falsch ist:

\begin{shaded}
\textbf{Axiom der Mikroaffinität.}
Sei~$\Delta := \{ \varepsilon \in \RR \,|\, \varepsilon^2 = 0 \}$ die
\emph{infinitesimale Monade} um~$0$. Sei~$f : \Delta \to \RR$ eine beliebige
Abbildung. Dann gibt es gewisse eindeutig bestimmte reelle Zahlen~$a$ und~$b$,
sodass für alle Zahlen~$\varepsilon$ in~$\Delta$ folgende Gleichung gilt:
\[ f(\varepsilon) = a + b \varepsilon. \]
\end{shaded}

Wir werden etwas Zeit benötigen, um zunächst die Aussage dieses Axioms zu
verstehen und dann seine Konsequenzen zu überblicken.


\section{Motivation für infinitesimale Zahlen}

In Abbildung~\ref{fig:schnittverhalten} sind zwei Situationen skizziert, bei
denen sich jeweils zwei Kurven schneiden. (Gerade Linien zählen
auch als \emph{Kurven}.) Es ist offensichtlich, dass sich im ersten
Bild die beiden Geraden in genau einem Punkt schneiden. Die Schnittsituation
beim zweiten Bild ist dagegen weniger klar. In klassischer Mathematik konstatiert man,
der Schnitt bestehe ebenfalls aus nur genau einem Punkt, nämlich dem Ursprung.
In der Tat könnte man keinen weiteren Punkt benennen, der ebenfalls auf beiden
Kurven liegen würde. Anschaulich scheint es aber ja doch einen Unterschied zu
geben -- man benötigt infinitesimale Zahlen, um ihn auf direkte Art und Weise
mathematisch einzufangen.\footnote{Indirekt geht es auch in klassischer
Mathematik: Die Parabel hat bei der Stelle~$x = 0$ eine \emph{doppelte
Nullstelle}. Das bedeutet, dass nicht nur der Funktionswert dort Null ist,
sondern auch noch seine erste Ableitung.}

\begin{aufgabeShaded}{Schnittberechnung}
Die Gleichungen der beiden Kurven im ersten Bild sind
\begin{empheq}[left=\empheqlbrace\ ]{align}
  y &= 2x, \\
  y &= 0.
\end{empheq}
Die erste Gleichung gehört zur schrägen Gerade, die zweite zur~$x$-Achse (auf
der alle Punkte als~$y$-Koordinate Null haben). Die Gleichungen der Kurven im
zweiten Bild sind
\begin{empheq}[left=\empheqlbrace\ ]{align}
  y &= x^2, \\
  y &= 0.
\end{empheq}
\begin{enumerate}
\item Löse das erste Gleichungssystem, um zu beweisen: Der einzige
Schnittpunkt~$(x|y)$ hat die Koordinaten~$x = 0$ und~$y = 0$. Wieso entspricht
das Schneiden der beiden Kurven rechnerisch der Lösungsmenge des kombinierten
Gleichungssystems?
\item Löse das zweite Gleichungssystem, um zu beweisen, dass ein Punkt~$(x|y)$
genau dann im Schnittbereich des zweiten Bilds liegt, wenn seine~$x$-Koordinate
Null und seine~$y$-Koordinate eine infinitesimale Zahl ist (also~$y^2 = 0$
erfüllt).

Hierbei ist es wichtig, mit Absicht langsam zu rechnen, um nicht durch zu
schnelles Vereinfachen die Pointe vorwegzunehmen. In klassischer Mathematik
gilt die Regel "`wenn~$y^2 = 0$, dann auch~$y = 0$''; erst mit dieser Regel
vereinfacht sich das Ergebnis, zu demselben wie in Teilaufgabe~a).
\end{enumerate}

\end{aufgabeShaded}

\end{document}

- Motivation: kleine Größen in der Physik
- Motivation: Ableitungsberechnung
- Mengenschreibweise (benötigt für Delta)
- Abbildungen
- "Schwebezustand" von Delta: Es stimmt nicht, dass es nur die Null enthalten
  würde -- andere Elemente kann man aber trotzdem nicht angeben.
