\documentclass[a4paper,ngerman,landscape,16pt]{scrartcl}
\usepackage[utf8]{inputenc}
\usepackage[ngerman]{babel}
\RequirePackage{amsmath,amsthm,amssymb,amscd,color,graphicx,environ}
\newcommand{\sur}[2]{\{ #1 \mathrel{|} #2 \}}
\definecolor{hicolor}{rgb}{.55,.12,.55}
\newcommand{\hil}[1]{\textcolor{hicolor}{#1}}
\usepackage{geometry}
\geometry{tmargin=1.5cm,bmargin=0.8cm,lmargin=3cm,rmargin=3cm}
%\geometry{tmargin=2cm,bmargin=3cm,lmargin=2cm,rmargin=2cm}
%\usepackage{lscape}
%\newcommand{\head}[1]{\section*{\rmfamily #1}}%begin{center}\large \textbf{#1}\end{center}}
%\let\raggedsection\centering
%\newcommand{\fuzzy}{\mathrel{||}}
\newenvironment{block}[1]{
  \begin{center}\textbf{#1}\end{center}
}{}
\begin{document}

\thispagestyle{empty}

\sffamily

\begin{block}{Regeln für surreale Zahlen}
\renewcommand{\labelenumi}{\arabic{enumi}.}
\begin{enumerate}
\item \emph{Konstruktionsprinzip.}
Sind~$L$ und~$R$ Mengen surrealer Zahlen und \hil{\textbf{ist kein Element von~$L$
$\geq$ irgendeinem Element von~$R$}}, so ist~$\sur{L}{R}$ ebenfalls eine surreale
Zahl. Alle surrealen Zahlen entstehen auf diese Art.

\item \emph{Notation.}
Für~$x = \sur{L}{R}$ bezeichnen wir ein typisches Element von~$L$
mit~"`$x^L$"', ein typisches Element von~$R$ mit~"`$x^R$"'. Wenn
wir~"`$\sur{a,b,c,\ldots}{d,e,f,\ldots}$"' schreiben, meinen wir die
Zahl~$\sur{L}{R}$, sodass~$a,b,c,\ldots$ die typischen Elemente von~$L$
und~$d,e,f,\ldots$ die typischen Elemente von~$R$ sind.

\item \emph{Anordnung.}

Wir sagen genau dann~$x \geq y$, falls kein $x^R \leq y$ und~$x \leq$
keinem $y^L$.

Wir sagen genau dann~$x \not\leq y$, wenn~$x \leq y$ nicht gilt.

Wir sagen genau dann~$x < y$, wenn $x \leq y$ und~$y \not\leq x$.

Wir sagen genau dann~$x \leq y$, wenn~$y \geq x$.

Wir sagen genau dann~$x > y$, wenn~$y < x$.

\item \emph{Gleichheit.}
Wir sagen genau dann~$x = y$, wenn~$x \leq y$ und~$y \leq x$.

\item \emph{Rechenoperationen.}
\begin{align*}
  x + y &:= \sur{x^L + y,\ x + y^L}{x^R + y,\ x + y^R}. \\
  -x &:= \sur{-x^R}{-x^L}. \\
  x - y &:= x + (-y). \\
  xy &:= \sur{x^Ly + xy^L - x^Ly^L,\ x^Ry + xy^R - x^Ry^R}{\\&\qquad\qquad\qquad x^Ly + xy^R -
x^Ly^R,\ x^Ry + xy^L - x^Ry^L}.
\end{align*}
\end{enumerate}
\end{block}

\end{document}
